\newglossaryentry{autot}
{
	name=Autotool,
	description={ist ein webbasiertes E-Learning-System, welches in der Lehre zum unterrichtsbegleitendem Stellen von Aufgaben verwendet wird. Eingesendete Lösungsversuche werden automatisch bewertet.}
}
\newglossaryentry{elear}
{
	name=E-Learning,
	description={(electronic learning) umfasst jegliche Formen des Lernens, bei denen elektronische Medien aus didaktischen Gründen verwendet werden.}
}
\newglossaryentry{hashadress}
{
	name=Hash-Adresse,
	description={ist der Wert, den eine Hash-Funktion einem Schlüssel zuordnet.}
}
\newglossaryentry{hashfunction}
{
	name=Hash-Funktion,
	description={ist eine Abbildung, welche eine Menge von Schlüsseln auf die Felder einer Hash-Tabelle abbildet.}
}
\newglossaryentry{hashtable}
{
	name=Hash-Tabelle,
	description={ist ein konkreter Datentyp, die das Wörterbuchproblem löst. Durch den Aufbau des konkreten Datentypen wird bei der Anwendung von Operationen ein direkter Zugriff ermöglicht.}
}
\newglossaryentry{haskl}
{
	name=Haskell,
	description={ist eine rein funktionale Programmiersprache mit einem statischen, starken Typsystem.}
}
\newglossaryentry{kollisione}
{
	name=Kollision,
	description={tritt zwischen zwei unterschiedlichen Schlüsseln auf, wenn diesen auf die gleiche Hash-Adresse abgebildet werden.}
}
\newglossaryentry{refac}
{
	name=Refactoring,
	description={ist die Verbesserung der Struktur eines Programmes, ohne dass sich dessen Verhalten nach außen verändert.}
}
\newglossaryentry{rehash}
{
	name=Re-Hashing,
	description={ist das Vergrößern oder Verkleinern einer Hash-Tabelle, in die danach alle zuvor beinhalteten Schlüssel neu eingefügt werden.}
}
\newglossaryentry{sondfolg}
{
	name=Sondierungsfolge,
	description={ist eine Folge von Hash-Adressen, die abhängig von einem Schlüssel beim geschlossenen Hashing mit Hilfe der Sondierungsfunktion berechnet wird.}
}