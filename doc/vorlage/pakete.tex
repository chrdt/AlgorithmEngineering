%% Definition der Codierung des Editors (sollte mit Editor übereinstimmen)
\usepackage[utf8]{inputenc} % verwende utf8 oder latin1
\usepackage{pgfplots}
%% Definition der Fonts in westeuropäischer Codierung
\usepackage[T1]{fontenc} % zur verbesserten Wortrennung
\usepackage{setspace}
%% Ändern der Spracheinstellung von LaTex auf deutsch 
\usepackage[ngerman]{babel}
\usepackage{float} 
\usepackage{pdflscape}
\usepackage{amssymb}
\makeatletter 
\renewcommand\paragraph{\@startsection{paragraph}{4}{\z@}% 
	{-3.25ex\@plus -1ex \@minus -.2ex}% 
	{1.5ex \@plus .2ex}% 
	{\normalfont\normalsize\bfseries}} 
\makeatother
%Zur Verwendung von Acronymen
\usepackage{acronym}
%% Verbesserter Zeilenumbruch durch Veränderung der Buchstabgröße
%\usepackage{microtype}

%% Korrektes Setzen fremdsprachlicher Anführungszeichen
\usepackage{csquotes}

%% Automatische Verlinkungen
\usepackage{hyperref}

%% Tabellen für wissenschaftliche Publikationen
\usepackage{booktabs}
% Mehrseitige Tabellen
%\usepackage{longtable}
\usepackage{color}
\usepackage{transparent}
%% Mathe-Pakete
\usepackage{amsmath} % Wichtige Mathematikfunktionalität (z.B. align)
%\usepackage{amstext} % Falls \text{} in Mathe-Umgebung nicht gut formatiert wird
\usepackage{amsfonts} % andere Schriftwarten für Mathe-Umgebungen
\usepackage{amssymb} % andere Symbole für Mathe-Umgebungen

%% Grafiken
\usepackage{graphicx} % für \includegraphics[]{}

% Parameter H für erzwingen von float-Positionen
%\usepackage{float}

% Tikz/PGF
\usepackage{tikz}
%% In SEW-MikTex Package tikzscale nicht vorhanden
%\usepackage{tikzscale} % Ermöglicht Einbinden und Skalierung von .tikz-Dateien mit \includegraphics
%\usetikzlibrary{external} % Ermöglicht vorkompilieren von tikz auf pdf
%\tikzexternalize[prefix=tikzexternal/] % TeXLive: -shell-escape MikTex: -enable-write18, prefix=Speicherordner für vorkompilierte PDFs
% Bestimmte libraries einbinden
%\usetikzlibrary{patterns}

%% Literaturverzeichnis
%% biblatex Konfiguration
\usepackage[
style=alphabetic, % gemeinsamer Befehl für citestyle und bibstyle (numeric,alphabetic)
% Nur verwenden wenn unterschiedlicher citestyle und bibstyle notwendig
%citestyle=alphabetic, % Aussehen des \cite-Befels (numeric,alphabetic)
%bibstyle=alphabetic, % Aussehen des \printbibliography-Befels (numeric,alphabetic)
%doi=false,isbn=false,url=false,eprint=false,% Unterdrücken von jew. Angaben in Literaturverzeichnis
%giveninits=true, % Vornamen abkürzen
backref=true, % Rückverlinkung in Literaturliste, benötigt hyperref
hyperref, % Hyperlinks in Zitaten, benötigt hyperref
%maxcitenames=1,% Max. Anzahl von Autorennamen in \cite
%maxbibnames=3,% Max. Anzahl von Autorennamen in Literaturverzeichnis
%uniquelist=minyear, % ermöglicht eine unterscheidung der Labels, wenn sie durch die Vorigen Befehle uneindeutig geworden sind
sorting=none,
backend=biber% generell nicht verändern
]{biblatex}

%% Verzeichnis der .bib-Datei
\addbibresource{literatur/literaturverzeichnis.bib}

%% Sonstige Einstellungen

%%%bibliography clean-up
%\AtEveryBibitem{% Clean up the reference list rather than editing the entries
%	\clearlist{address}
%	\clearlist{location}
%	\clearfield{date}
%	\clearfield{month}
%	\clearfield{series}
%	
%	\ifentrytype{book}{}{% Remove publisher and editor except for books
%		\clearlist{publisher}
%		\clearname{editor}
%	}
%}
%% END bibliography clean-up
%
%% BEGIN citation clean-up (e.g. for full citations of own publications in project reports)
%\AtEveryCitekey{% Clean up citations rather than editing the entries
%	\clearname{editor}%
%}
%% END citation clean-up
%
%% BEGIN squeeze space in bibliography by replacing strings
%\DeclareSourcemap{ 
%	\maps[datatype=bibtex]{% If you have to squeeze space do it here not in the entries!
%		\map{
%			\step[fieldsource=series, match=Lecture Notes in Computer Science, replace=LNCS]
%			\step[fieldsource=publisher, match=\regexp{Gesellschaft .* Informatik}, replace=GI]
%			\step[fieldsource=publisher, match=\regexp{ \(GI\)}, replace=]           
%		}
%	}      
%}
%% END squeeze space in bibliography by replacing strings

%% Quellcode einbinden
\usepackage{listings}
%\usepackage{mdframed} % automatische Seitenumbrüche bei Boxen
\usepackage{xcolor}
% Definitionen für Listing
%\definecolor{mygreen}{rgb}{0.133,0.545,0.133}
%\definecolor{mypurple}{rgb}{0.627,0.126,0.941}
%\lstset
%{
%	language=Matlab, 
%	breaklines=true,
%	basicstyle=\ttfamily,
%	showstringspaces=false,
%	captionpos=t,
%	keywordstyle=\color{blue}\bfseries\sffamily,
%	commentstyle=\color{mygreen}\slshape,
%	stringstyle=\color{mypurple},
%	prebreak=\mbox{$\hookleftarrow$},
%	belowcaptionskip=1em
%}

% Geometry: Nur wenn Satzspiegelberechnung von KOMA-Script missachtet werden soll
%\usepackage{geometry}

% Einheiten, lädt automatisch nicefrac
%\usepackage{units}

%\usepackage{lipsum}